\documentclass[letterpaper, 12pt]{article}
\usepackage[utf8]{inputenc}
\usepackage[margin=1.0in]{geometry}
%\usepackage{showframe}
\newcommand*{\SignatureAndDate}[1]{%
    \par\noindent\makebox[2.5in]{\hrulefill} \hfill\makebox[2.0in]{\hrulefill}%
    \par\noindent\makebox[2.5in][l]{#1}      \hfill\makebox[2.0in][l]{Date}%
}%
% Specialized Sections
% Top Section
\renewcommand\thesection{\Roman{section}}
% Subsections
\renewcommand\thesubsection{\Alph{subsection}}


\title{Constitution for the Olivet Association for Computing Machinery (OACM)}
\author{OACM}
\date{\today}

\begin{document}
\maketitle
\section{Name}
Olivet Association for Computing Machinery (OACM)
\section{Purpose}
To provide technological education, assistance, and support for its members, the college, and the surrounding community.
\section{Membership}
\begin{enumerate}
    \item \textit{Olivet College Compact.} Each individual member of the OACM 
        must embrace the Olivet College Compact principles: value diversity 
        within a community built on trust, participation and a sense of pride.  
    \item \textit{The ACM Code of Ethics.}  Each member of the OACM must 
        abide by and embrace the ACM Code of Ethics and Professional Conduct 
        in addition to the compact. \\
        http://www.acm.org/about-acm/acm-code-of-ethics-and-professional-conduct
    \item \textit{Qualifications:}
        \begin{enumerate}
            \item Membership will be granted to any enrolled student who 
                indicates an interest in OACM.  No prior experience in 
                computing, computer science, or anything in the realm of is 
                required.
            \item All members are required to become registered as Student 
                members of the national association of the ACM.
        \end{enumerate}
     \item \textit{The club shall not discriminate against any human being 
         other then the qualifications above.} We open our organizational 
         arms equally wide to people from every race, gender, gender 
         identity, sexual orientation, size, ethnic origin, location of 
         habitation, religion, external appearance, national origin, age, 
         sex, height, weight, familial status, or marital status, and 
         anything that could possibly ever be used as a basis of 
         discrimination. 
\end{enumerate}
\section{Objectives of Organization}
The OACM maintains three solid and unwaivering goals 
\begin{enumerate}
    \item Education
        \begin{enumerate}
            \item The OACM is committed to further the education of its 
                members from the expansion of known languages, programming 
                techniques, to solidifying/creating a programmatic base for 
                the member.
                In addition, the OACM shall make its goal to, spread the 
                awareness and promotion of the fair treatment and possible 
                existence of any and all Artificial Intelligence.
        \end{enumerate}
    \item Outreach
        \begin{enumerate}
            \item The OACM is dedicated to the community outreach both on 
                campus and off campus. The group encourages, embraces, and 
                seeks out chances to improve the programmatic knowledge and 
                technological prowess of the local and surrounding 
                communities through both training and on site assistance.
        \end{enumerate}
    \item Preperation
        \begin{enumerate}
            \item The OACM is committed to helping its members be prepared 
                for the after undergraduate market by encouraging 
                opportunities such as internships, Research Experience for 
                Undergraduatess, and the like in any form which the group 
                can provide. 
                Whether its connections, suggestions, or workshops.
        \end{enumerate}
\end{enumerate}
\section{Governance}
The governing body of the OACM shall be the OACM's Officers.
\begin{enumerate}
    \item It shall be the duty of the Chair to administer all of the business 
        related to the overall welfare of the Student Organization. 
        The Chair and Vice Chair will also compile and enforce regulations 
        governing the OACM, which do not violate the policies established by 
        Olivet College.
    \item It shall be recognized that the club's primary responsibilities are directed towards the benefit of the Students. 
\end{enumerate}
\section{Officers}
\subsection{Number and Term}
The Executive Council of the OACM shall be comprised of the following offices:
\begin{enumerate}
    \item \textbf{Chair}
        \begin{enumerate}
            \item Limited to 2 year term limit beginning from the fall after 
                initial election.
            \item Minimum 2 year membership to become eligible or 1 year as 
                Vice Chair. Exception only made in event no member meets 
                these criteria.
        \end{enumerate}
    \item \textbf{Vice-Chair}
        \begin{enumerate}
            \item Limited to 3 year term limit beginning from the fall after 
                initial election.
            \item Minimum 1 year membership to become eligible. 
        \end{enumerate}
    \item \textbf{Treasurer}
        \begin{enumerate}
            \item Limited to 4 year term limit beginning from the fall after 
                initial election.
            \item No year minimum.
        \end{enumerate}
    \item \textbf{Secretary}
        \begin{enumerate}
            \item No term limit.
            \item No year minimum.
        \end{enumerate}
    \item \textbf{SGA Representative}
        \begin{enumerate}
            \item No term limit.
            \item No year minimum.
        \end{enumerate}
\end{enumerate}
\subsection{Duties of the Officers}
\begin{enumerate}
    \item \textbf{Chair.}  It shall be the duty of the Chair to conduct all 
        meetings, to lead discussions therein, to recognize all speakers, 
        and to see that all meetings are carried on in accordance with 
        parliamentary procedure.  
        It shall be the duty of the Chair to call all special meetings 
        and to act as the presiding officer.
    \item \textbf{Vice Chair.} It shall be the duty of the Vice Chair to act 
        as an aide to the Chair and to carry on the duties of the Chair in 
        event of his or her absence.  
        The Vice Chair handles duties that would otherwise handled by the 
        Chair if possible. 
        Any tasks that would not be completed by the Chair would be completed 
        by the Vice Chair. 
        The Vice Chair is allowed to represent the OACM in a formal setting.
    \item \textbf{Treasurer.} It shall be the duty of the Treasurer to 
        handle any incidental funds the OACM may have in its control at the 
        discretion of the Chair and Vice Chair.
    \item \textbf{Secretary.} It shall be the duty of the Secretary to keep 
        written minutes of all meetings and to inform the members of the 
        Council of the proceedings.  
        The Secretary shall also keep on file a written record of all of the 
        deliberations of the Executive Council.  
        It shall also be the duty of the Secretary to receive and approve or 
        disapprove of absences and excuses.
    \item \textbf{SGA Representative.} It shall be the duty of the SGA 
        Representative to be present at all SGA meetings and sign the OACM 
        into these meetings. 
        The SGA Rep shall also take relevant notes of the meetings and 
        provide them to the remaining council members.
\end{enumerate}
\subsection{Selection of Officers}
\begin{enumerate}
    \item Nominations
        \begin{enumerate}
            \item The position of Chair and Vice Chair may be nominated by 
                officers though suggestion from other members is welcomed 
                and considered.
            \item Any member can nominate any other member, who fit the 
                required criteria, for any of the remaining positions.
        \end{enumerate}
    \item Elections
        \begin{enumerate}
            \item Elections shall be held in an open forum style where 
                officers speak first and ask for nominations.
            \item Elections shall proceed in the following order:
                \begin{enumerate}
                    \item Chair
                    \item Vice Chair
                    \item Treasurer
                    \item Secretary
                    \item SGA Representative
                \end{enumerate}
            \item Those who are nominated and do not meet the minimum year 
                requirements are immediately disqualified for running unless 
                given explicit approval from the other officers otherwise 
                all officers shall be elected by a majority vote of the 
                members.
        \end{enumerate}
\end{enumerate}
\section{Resignation of Officers}
In the event that an OACM member or alternate should choose to resign from 
office, that member of alternated must submit an formal or informal letter 
of resignation to the Chair or Vice Chair.  
At the next meeting, the Chair shall handle the resignation under New 
Business.  
The subsequent vacancy shall be filled as outlined in the Constitution. 
\section{Impeachment of Officers}
In order for an OACM officer to be impeached from office, four letters from 
current officers must be submitted to the Chair (or Vice Chair in the case 
of a Chair impeachment).  
At the next meeting, the Chair shall handle the impeachment under New 
Business (vice Chair in the case of a Chair impeachment).  
The Executive Council shall entertain a question and answer period with the 
officer in question.  
Witnesses may be present, with the approval of the Council.  
Following the question and answer period, the officer in questions shall be 
excused from the meeting and a discussion period shall ensue.  
Following the discussion the impeachment shall be put to a vote.  
In order for an impeachment to carry, they must accept the impeachment by 
two-thirds (2/3) vote of the remaining OACM officers.  
The subsequent vacancy shall be filled as outlined in Section in the 
Constitution.
\section{Advisor}
\begin{enumerate}
    \item The Collegiate Advisor 
        \begin{enumerate}
            \item Must be a member of the faculty or staff of Olivet 
                College.
            \item He/She maintains continuity, offers guidance, advises as 
                to college procedures serves as a sounding board, assists 
                officers, and is an all-around resource person.
            \item The term of office for the advisor is one year, renewable 
                indefinitely.
        \end{enumerate}
    \item The Professional Advisor
        \begin{enumerate}
            \item required by the national ACM, must not be a member of the 
                Olivet College however this is heavily advised.
            \item This position is for a year renewable indefinitely.
        \end{enumerate}
    \item The Professional Advisor and the Collegiate Advisor are required 
        to be filled before each academic year begins.
        Though the Professional Advisor does not also have to fill the 
        Collegiate Advisor position or vice-versa.
\end{enumerate}
\section{Meetings}
\begin{enumerate} 
    \item Types of Meetings
        \begin{enumerate}
            \item Regular meetings will be held once biweekly during the 
                academic year.
            \item Special meetings are defined as those meetings that take 
                place outside of regularly scheduled meetings.  
        \end{enumerate}
    \item Location and frequency of meetings
        \begin{enumerate}
            \item Meetings will normally be held in (Mott, Room 112/303)
            \item Meetings will normally occur (Wednesday 10:10-10:40)
        \end{enumerate}
    \item Who May Call Meetings
        \begin{enumerate}
            \item Regular meetings are called by the Chair and are called 
                based upon a schedule determined at the beginning of each 
                term/semester.
            \item Special meetings may be called by the Chair or at the 
                request of three members.
        \end{enumerate}
\end{enumerate}
\section{Ratification of the Constitution}
\begin{enumerate} 
    \item Adoption
        \begin{enumerate} 
            \item This constitution must be ratified by a two-thirds 
                majority of the members or by the Chair and Vice Chair 
                during the 2016-2017 academic year.
            \item The Constitution will take effect once approved by the 
                Advisor.
        \end{enumerate}
    \item Amendments
        \begin{enumerate}
            \item During the 2016-2017 year, the constitution may be amended 
                at the will of the Chair and Vice Chair, only if they agree 
                upon the amendment. 
                This amended can be contested by members and vetoed by 
                a \(\frac{1}{2} + 1\) vote by the general members. 
                Members will be informed of changes done in this way.
            \item Any member may propose an amendment.  
                This proposed amendment must be in writing and presented at 
                a regular meeting.
            \item Members must be notified that a vote will take place on an 
                amendment at least one week prior to the meeting at which 
                the vote is to occur.
            \item A voted on amendment will be taken at the next regular 
                meeting to allow the membership to discuss and/or debate the 
                pro and cons of the amendment.  
                A two-thirds majority of the total membership is required 
                for passage.  
            \item The constitution will take effect once approved by the 
                Advisor. 
        \end{enumerate}
\end{enumerate}
\section{Signatures}
\vspace{.2in}

\SignatureAndDate{President of Organization}
\vspace{.2in}

\SignatureAndDate{Advisor of Organization}
\vspace{.2in}

\SignatureAndDate{Director of Student Clubs and Organizations}
\vspace{.2in}

\SignatureAndDate{VP of Student Life}
\vspace{.2in}


Ratified 9 - 20 - 2016
\end{document}